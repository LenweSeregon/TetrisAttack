\documentclass[a4paper,utf8]{article}
\usepackage{exemple}
\usepackage[normalem]{ulem}
\usepackage{amsfonts}
\usepackage{graphicx}
\usepackage{MnSymbol,wasysym}
\usepackage{hyperref}
\usepackage[french]{babel}

\formation{L3MI}
\date{20/02/2017}
\matiere{Conception Orient\'ee Objet}
\titre{Tetris Attack - Rapport 2 - Avancement}

\newcommand\code[1]{\textsf{#1}}
\newcommand\srdjan[1]{{\color{red} #1}}

\begin{document}

\entete

\section{Ce qui a \'et\'e fait et reste \`a faire dans la r\'epartion actuelle des t\^{a}ches}
	\subsection{Charlotte}
		Charlotte s'occupe de la partie principale du menu, des cr\'edits et de l'aide.
		\subsubsection{Travail r\'ealis\'e}
			\begin{itemize}
				\item La fen\^{e}tre du menu en lui m\^{e}me contenant chacun des \'el\'ements demand\'es
				\item Le d\'eplacement parmis les boutons avec les fl\`eches directionnelles haut et bas
				\item Chaque bouton est fonctionnel, choisir l'un d'eux d\'eclenche la bonne action (pour le moment, ouverture d'une nouvelle JFrame)
				\item Le design du menu est bien avanc\'e
			\end{itemize}
		\subsubsection{Travail \`a r\'ealiser ensuite}
			\begin{itemize}
				\item Supprimer le mouseListener
				\item R\'egler un soucis de touche
				\item Finir le design
				\item UML
				\item Rajouter une musique
			\end{itemize}
	\subsection{Antoine}
		Antoine s'occupe du menu des options et de la gestion des changements inh\'erents aux modifications de celles-ci.
		\subsubsection{Travail r\'ealis\'e}
			\begin{itemize}
				\item La fen\^{e}tre des options en elle-m\^{e}me
				\item Le d\'eplacement parmi les boutons avec les fl\`eches directionnelles haut et bas
				\item Impl\'ementation des options (commandes joueur 1, joueur 2 et son)
				\item Le design du menu est bien avanc\'e
			\end{itemize}
		\subsubsection{Travail \`a r\'ealiser ensuite}
			\begin{itemize}
				\item Supprimer l'actionListener
				\item R\'egler un soucis de focus
				\item Finir le design
				\item UML
				\item Rajouter une musique
				\item Rajouter l'option de redimensionnage 
			\end{itemize}
	\subsection{Bertrand}
		Bertrand s'occupe de l'acceuil, du design du jeu et de la gestion des ressources.
		\subsubsection{Travail r\'ealis\'e}
			\begin{itemize}
				\item La fen\^{e}tre de l'acceuil en elle-m\^{e}me
				\item Une des animations (le "push any key")
				\item Ajout de musique
			\end{itemize}
		\subsubsection{Travail \`a r\'ealiser ensuite}
			\begin{itemize}
				\item Acceuil :
					\begin{itemize}
						\item Lancer le menu et non le jeu quand on appuie sur une touche
						\item Supprimer le gif de fond et r\'ealiser une vraie animation
						\item Redimensionner
					\end{itemize}
				\item Design: finir le design d\'ej\`a entam\'e
				\item Ressources : une fois tous les designs finis, faire le m\'enage de toutes les ressources inutilis\'ees
				\item UML
			\end{itemize}
	\subsection{Nicolas}
		Nicolas s'occupe du gameplay.
		\subsubsection{Travail r\'ealis\'e}
			\begin{itemize}
				\item La fen\^{e}tre du jeu
				\item D\'eplacement de chacun des joueurs dans sa propre grille
				\item Possibilit\'e de swap deux cases
				\item Gestion de la d\'etection de suite de cases
				\item Timer
				\item Apparition de blocs toutes les x secondes
				\item D\'etection de victoire/defaite
			\end{itemize}
		\subsubsection{Travail \`a r\'ealiser ensuite}
			\begin{itemize}
				\item R\'egler quelques bug signal\'es sur slack
				\item Faire appara\^{i}tre petit \`a petit la ligne en bas
				\item Rajouter une vraie action en cas de victoire
				\item Apparition de blocs "mur" chez l'adversaire en cas de coup sp\'ecial
				\item Affichage des scores
				\item IA
			\end{itemize}


\section{Apr\`es tout \c{c}a}
	Une fois tout ce travail r\'ealis\'e, nous aurons d\'ej\`a une application bien fonctionnelle. Cependant, nous n'aurons l\`a qu'une version basique du jeu ; l'impl\'ementation des am\'eliorations, ajouts et id\'ees du rapport n\°1 ne sont encore assign\'e \`a aucun d'entre nous, nous verrons en fonction du temps restant. Ci celui-ci nous le permet, quand l'un de nous aura fini \`a 100\% sa partie, certaines de ces options seront s\'erieusement envisag\'ees. Pour rappel :
	\begin{itemize}
		\item Si le joueur r\'ealise une forme particuli\`ere avec ses blocs (peut-\^{e}tre une forme de banane), une action sp\'ecifique est lanc\'ee. Celle-ci pourrait \^{e}tre une sorte d'easter egg qui pourrait \^{e}tre l'apparition d'une banane qui, si on arrive \`a l'utiliser, pourrait provoquer une assez grosse explosion sur la grille du joueur.
		\item Possibilit\'e aussi dans le m\^{e}me style : des actions particuli\`eres pourrait d\'eclencher des petites attaques contre l'adversaire, par exemple envoyer une banane qui en explosant bloquerait temporairement des cases, ou rendrait une zone "glissante" dans le sens o\`u il y serait plus compliqu\'e d'y bouger les cases \`a l'endroit voulu.
	\end{itemize}


\end{document}
